\section{形容詞}%

\subsection{形容词的分类}%

\noindent 属性形容詞:表示事物的客观属性、状态。

\noindent 感情形容詞:表示人的喜怒哀乐、爱憎好恶等感情、心理感受,或是疼痛、冷热等身体感觉。


\subsection{形容词的活用}%

\begin{table}[h]
  \centering
  \caption{形容词活用示例}
  \begin{tabular}{c | c | c c c c c}
    词例 & 词干 & 未然形 & 连用形 & 终止形 & 连体形 & 假定形 \\
    \hline
    暑い & 暑 & かろ & \makecell{\cn[1] く \\ \cn[2] かっ} & い & い & けれ \\
  \end{tabular}
\end{table}


\subsubsection{未然形}%

后接推测助动词「う」,构成简体推量形式,表示推测。
\begin{itemize}
  \item 明日も 暑かろう。
\end{itemize}

\subsubsection{连用形「く」}

\begin{enumerate}
  \item 置于所修饰用言前作状语。
    \begin{itemize}
      \item 速く 食べます。
    \end{itemize}
  \item 与「なる」或「する」结合表示变化。
    此时,「形容词连用形+なる」可以看作一个自动词,表示客观变化。
    「形容词连用形+する」可以看作一个他动词,表示人为地改变。
    \begin{itemize}
      \item 人口が增え、市場も大きくなった。
    \end{itemize}
  \item 两个用言连用时表示中顿。
    \begin{itemize}
      \item この部屋は 広くて明るい。
    \end{itemize}
  \item 后接补助形容词「ない」表示否定。
    \begin{itemize}
      \item 今日の天気は よくない。
    \end{itemize}
\end{enumerate}


\subsubsection{连用形「かっ」}%

后接过去完了助动词「た」。
\begin{itemize}
  \item 昨日の天気は よかった。
\end{itemize}


\subsubsection{终止形}%

\begin{enumerate}
  \item 作谓语结句。
  \item 后接助词「から」、「けれども」、「し」、「か」、「よ」、「ね」等。
\end{enumerate}


\subsubsection{连体形}%

后接体言作定语。
\begin{itemize}
  \item 面白い映画を みます。
\end{itemize}


\subsubsection{假定形}%

后接接续助词「ば」,表示假定条件。
\begin{itemize}
  \item 美味しければ 食べます。
\end{itemize}


\subsection{補助形容詞}%

接在其它用言后起补助作用的形容词,包括「ない」、「ほしい」、「いい」。

{\bf
\noindent ない
}

接在形容词、形容动词、形容词型助词、形容词型助动词的连用形之后,
表示否定。

{\bf
\noindent ほしい
}

接在「用言连用形+て」或「用言未然形+ないで」后,
表示说话人希望出现某一事态。

{\bf
\noindent いい
}


