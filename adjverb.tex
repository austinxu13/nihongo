\section{形容動詞}%

\subsection{形容动词的分类}%

\noindent 属性形容動詞。

\noindent 感情形容動詞。


\subsection{形容动词的活用}%

\begin{table}[h]
  \centering
  \caption{形容动词活用示例}
  \begin{tabular}{c | c | c c c c c}
    词例 & 词干 & 未然形 & 连用形 & 终止形 & 连体形 & 假定形 \\
    \hline
    静かだ & 静か & だろ & \makecell{\cn[1] で \\ \cn[2] に \\ \cn[3] だっ} & だ & な & なら \\
  \end{tabular}
\end{table}

{\bf
\noindent 未然形
}

后接推测助动词「う」,表示推测。
\begin{itemize}
  \item 駅前は 夜も にぎやかだろう。
\end{itemize}

{\bf
\noindent 连用形「で」
}

置于所修饰用言前作状语。
\begin{itemize}
  \item 速く 食べます。
\end{itemize}

与「なる」或「する」结合表示变化。
此时,「形容词连用形+なる」可以看作一个自动词,表示客观变化。
「形容词连用形+する」可以看作一个他动词,表示人为地改变。
\begin{itemize}
  \item 人口が增え、市場も大きくなった。
\end{itemize}

两个用言连用时表示中顿。
\begin{itemize}
  \item この部屋は 静かで きれいです。
\end{itemize}

后接补助形容词「ない」表示否定。
\begin{itemize}
  \item あまり好きではないが、さほど嫌いでもない。
\end{itemize}

{\bf
\noindent 连用形「に」
}

修饰后续用言,作状语。
\begin{itemize}
  \item 部屋を きれいに 掃除しました。
\end{itemize}

与「なる」或「する」结合表示变化。
此时,「形容词连用形+なる」可以看作一个自动词,表示客观变化。
「形容词连用形+する」可以看作一个他动词,表示人为地改变。
\begin{itemize}
  \item 人口が增え、市場も大きくなった。
\end{itemize}

{\bf
\noindent 连用形「だっ」
}

后接过去完了助动词「た」。
\begin{itemize}
  \item 昨日は 暇だった。
\end{itemize}

{\bf
\noindent 终止形
}

作谓语结句。

后接助词「から」、「けれども」、「し」等。

{\bf
\noindent 连体形
}

后接体言作定语。
\begin{itemize}
  \item 図書館は 静かな所です。
\end{itemize}

{\bf
\noindent 假定形
}

后接接续助词「ば」,表示假定条件。
\begin{itemize}
  \item そこが 静かなら そこで 勉強します。
\end{itemize}

