\section{助詞}%
\label{sec:_}


%%%%%%%%%%%%%%%%%%%%
\subsection{格助詞}%
%%%%%%%%%%%%%%%%%%%%

{\bf
\noindent が
}

\begin{enumerate}
  \item 表示主语:銀行が あります。
\end{enumerate}

{\bf
\noindent を
}

\begin{enumerate}
  \item 表示宾语:私は よくインターネットをします。
\end{enumerate}

{\bf
\noindent の
}

\begin{enumerate}
  \item 表示定语:私はの名前は 陳紅です。
\end{enumerate}

{\bf
\noindent に
}

\begin{enumerate}
  \item 表示存在的场所:冷蔵庫に あります。
  \item 表示动作的时间:八時に 授業が あります。
  \item 表示比较的基准:運動は 体に いいです。\\ 
    弟は 一日に二時間 本を読みます。\\
    日本の若者にとって、留学は珍しいことではありません。
  \item 表示动作涉及的对象:両親に 電話します。
\end{enumerate}

{\bf
\noindent より
}

\begin{enumerate}
  \item 表示比较的基准:銀行は 郵便局より 近いです。
\end{enumerate}

{\bf
\noindent で
}

\begin{enumerate}
  \item 表示动作发生的场所:教室で 宿題をします。
  \item 表示进行动作的手段:万年筆で 手紙を書きます。
  \item 表示范围:日本では 富士山が いちぼん 高いです。
\end{enumerate}

{\bf
\noindent と
}

\begin{enumerate}
  \item 表示动作的共同者:友達と 昼ご飯を食べます。
  \item 表示思考、叙述的内容:私は彼が来ると思います。\\
    田中さんはお金を持っていると言いました。
\end{enumerate}

{\bf
\noindent から
}

\begin{enumerate}
  \item 表示动作的起点:朝から彼を待ている。
  \item 表示原因:用事があるから、明日の会議には出席できません。
\end{enumerate}

{\bf
\noindent まで
}

\begin{enumerate}
  \item 表示动作的终点:デパートは 何時までですか。
\end{enumerate}

{\bf
\noindent へ
}

\begin{enumerate}
  \item 表示动作的方向:今日 王さんは学校へ勉強に来ました。
  \item 表示动作的对象:これは母への手紙です。
\end{enumerate}



%%%%%%%%%%%%%%%%%%%%%%
\subsection{語気助詞}%
%%%%%%%%%%%%%%%%%%%%%%

{\bf
\noindent か
}

\begin{enumerate}
  \item 表示疑问:それは なんですか?
\end{enumerate}

{\bf
\noindent ね
}

\begin{enumerate}
  \item 表示感叹:王さんは 日本語が お上手ですね。
\end{enumerate}

{\bf
\noindent よ
}

\begin{enumerate}
  \item 表示判断,主张或提醒对方注意:明日、李さんも 行きますよ。
\end{enumerate}

{\bf
\noindent な
}

接在动词终止形之后,表示禁止:
\begin{enumerate}
  \item そんな冗談を言うな。
\end{enumerate}



%%%%%%%%%%%%%%%%%%%%%%
\subsection{並立助詞}%
%%%%%%%%%%%%%%%%%%%%%%

并列助词用来并列两个或两个以上的处于同等地位的内容词或词组,
构成联合式词组。
所构成的联合式词组,
不具备``格'',不能体现这个联合式词组在句中的地位及其与其他词的关系,
必须借助加在其后的格助词构成主宾补定语等成分,
或加助动词构成谓语成分。


{\bf
\noindent と
}

\begin{enumerate}
  \item 表示并列,列举事物的全部:私は 兄と姉が います。
\end{enumerate}


{\bf
\noindent や
}

\begin{enumerate}
  \item 表示列举,列举全体的一部分:郵便局には 雑誌や新聞が あります。
\end{enumerate}


{\bf
\noindent たり
}

接在用言连用形后,
表示动作在某一时间内交替进行或状态交替出现。
构成的联合式词组具有体言的性质,
还可以后接「する」像サ变动词词干使用,
或者直接如句作状语。

\begin{enumerate}
  \item 寒かったり暑かったり
\end{enumerate}


%%%%%%%%%%%%%%%%%%%%%%
\subsection{提示助詞}%
%%%%%%%%%%%%%%%%%%%%%%

{\bf
\noindent も
}

\begin{enumerate}
  \item 表示同类事物的追加:私も 一人っ子です。
  \item 表示强调:昨日の晩餐会には、お貴客さんが50人も来ました。
\end{enumerate}


{\bf
\noindent しか
}

接在体言后。
\begin{enumerate}
  \item 表示限定某一事物,对除此以外的一切事物都加以否定,与否定形式的谓语呼应:
    彼は小説しか読まない。\\
    教室には一人しかいない。
\end{enumerate}


%%%%%%%%%%%%%%%%%%%%%%
\subsection{副助詞}%
%%%%%%%%%%%%%%%%%%%%%%

副助词接在体言、相当于体言的词语后。
与其所附属的内容词作状语,修饰后面的用言。

{\bf
\noindent ほど
}

\begin{enumerate}
  \item 表示比较的基准:今日は 昨日ほど 忙しく ありません。
\end{enumerate}

{\bf
\noindent くらい 
}

\begin{enumerate}
  \item 表示概数:月に 三回くらい 電話をします。
\end{enumerate}

{\bf
\noindent か
}

接在体言后。
\begin{enumerate}
  \item 表示不确定的事物:何人かで行く。
\end{enumerate}

{\bf
\noindent だけ
}

接在体言、用言连体形后。
\begin{enumerate}
  \item 表示限定:あなただけが知っているでしょう。\\
    こプールは夏の間だけ開きます。
\end{enumerate}

{\bf
\noindent ごとに
}

接在体言后。
\begin{itemize}
  \item 相当于汉语``每······'': 三時間ごとに注射する。
\end{itemize}

{\bf
\noindent ずつ
}

接在表示数量、程度的体言和副词、助词后,
表示按同样的数量分配或同样的程度反复。
\begin{itemize}
  \item 毎朝一本ずつ牛乳を飲む。
  \item 新しい地下鉄や道路も少しずつ造っています。
\end{itemize}


%%%%%%%%%%%%%%%%%%%%%%
\subsection{接続助詞}%
%%%%%%%%%%%%%%%%%%%%%%

接续助词连接用言、用言性词组,
表示它们之间的关系,
在句子中起承上启下的作用。

{\bf
\noindent が
}

\begin{enumerate}
  \item 表示转折:勉強は 忙しいですが、楽しいです。
\end{enumerate}

{\bf
\noindent ので
}

接在用言、助动词的连体形后。
\begin{enumerate}
  \item 表示客观的因果关系:彼は若いので、元気があります。
\end{enumerate}

{\bf
\noindent て
}

接在动词、形容词、助动词的连用形之后。
\begin{enumerate}
  \item 表示并列:この川は長くて広いです。
  \item 表示动作的先后关系:夏は過ぎて秋が来た。
  \item 表示动作进行的方式:父は毎日電車に乗って会社へ行きます。
  \item 表示原因:熱があって、学校を休みました。
\end{enumerate}

{\bf
\noindent ても(でも)
}

「ても」接在动词、形容词、助动词的连用形之后,
「でも」接在名词、形容动词词干后,
表示让步条件的逆态接续,
相当于汉语``即使······也······''。
\begin{itemize}
  \item たとえ雨が降っても、行きます。
  \item 雪は夜になってもやみませんでした。
  \item それは子供でもできる問題です。
  \item スーパーマーケットでは何でも売っています。
\end{itemize}


{\bf
\noindent けれども
}

接在用言、助动词的终止形后。
\begin{itemize}
  \item 表示转折:話せるけれども、書けない。
\end{itemize}

{\bf
\noindent ながら
}

接在动词连用形后。
\begin{itemize}
  \item 表示动作同时进行:あの人はいつも新聞を読みながら、ご飯を食べます。
\end{itemize}


{\bf
\noindent し
}

接在用言终止形后。
\begin{itemize}
  \item 表示两个事项同时存在:この花は色も綺麗だし、香りもいいです。
\end{itemize}
