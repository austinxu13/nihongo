\section{助動詞}%

有形态变化的功能词。
主要接在以动词为主的用言后,
从时、态等方面对用言加以补充,
或增添各种表示陈述方式等的意义。

\subsection{构成敬体的助动词}%

\subsubsection{ます}%

接在动词和动词型助动词的连用形之后,
表示说话者和作者的一种郑重的心情,
表示对听者和读者的尊敬,
构成敬体。
广泛用于谈话中。

「ます」属于特殊活用。

\begin{table}[h]
  \centering
  \caption{「ます」活用示例}
  \begin{tabular}{c c c c c c c}
    基本形 & 未然形 & 连用形 & 终止形 & 连体形 & 假定形 & 命令形 \\
    \hline
    ます & \makecell{ませ \\ ましょ} & まし & ます & ます & ますれ & \makecell{ませ \\ まし} \\
  \end{tabular}
\end{table}

\begin{itemize}
  \item 私わ お茶を飲みません。
  \item みんなもう帰りました。
  \item 毎日、六時に起きます。
  \item いらっしゃいませ。
\end{itemize}


\subsubsection{です}%

接在形容词、形容词型助动词之后,
表示说话者和作者的一种郑重的心情,
表示对听者和读者的尊敬,
构成敬体。

「ます」属于特殊活用,一般只使用其未然形和终止形。

\begin{table}[h]
  \centering
  \caption{「です」活用示例}
  \begin{tabular}{c c c c c c c}
    基本形 & 未然形 & 终止形 \\
    \hline
    です & でしょ & です \\
  \end{tabular}
\end{table}



\subsection{表示各种陈述方式的助动词}%

\subsubsection{たい}%

接在动词的连用形之后,
表示说话人内心的愿望,
相当于汉语``想······''。

「たい」属于形容词型活用。

\begin{table}[h]
  \centering
  \caption{「たい」活用示例}
  \begin{tabular}{c | c | c c c c c}
    基本形 & 未然形 & 连用形 & 终止形 & 连体形 & 假定形 \\
    \hline
    たい & たかろ & \makecell{\cn[1] たく \\ \cn[2] たかっ} & たい & たい & たけれ \\
  \end{tabular}
\end{table}

\begin{itemize}
  \item 早く帰りたかろう。
  \item 私は 薬を飲みたくないです。
  \item 昨日、家へ帰りたかったです。
\end{itemize}




