\section{助動詞}%

有形态变化的功能词。
主要接在以动词为主的用言后,
从时、态等方面对用言加以补充,
或增添各种表示陈述方式等的意义。

\subsection{表示态的助动词}%

\subsubsection{被动助动词「れる」、「られる」}%

接在动词未然形后构成动词的被动态,属于下一段动词型活用。

\begin{table}[h]
  \centering
  \caption{「れる」、「られる」活用示例}
  \begin{tabular}{c c c c c c c}
    基本形 & 未然形 & 连用形 & 终止形 & 连体形 & 假定形 & 命令形 \\
    \hline
    れる & れ & れ & れる & れる & れれ & \makecell{れろ \\ れよ} \\
    られる & られ & られ & られる & られる & られれ & \makecell{られろ \\ られよ} \\
  \end{tabular}
\end{table}

\subsubsection{使动助动词「せる」、「させる」}%

接在动词未然形后构成动词的被使动态,属于下一段动词活用形。

\begin{table}[h]
  \centering
  \caption{「せる」、「させる」活用示例}
  \begin{tabular}{c c c c c c c}
    基本形 & 未然形 & 连用形 & 终止形 & 连体形 & 假定形 & 命令形 \\
    \hline
    せる & せ & せ & せる & せる & せれ & \makecell{せろ \\ せよ} \\
    させる & させ & させ & させる & させる & させれ & \makecell{させろ \\ させよ} \\
  \end{tabular}
\end{table}


\subsection{表示时的助动词}

过去完了助动词「た」,
主要用来构成过去时,
表示完成。
接在用言、助动词的连用形后。
「た」属于特殊活用。

\begin{table}[h]
  \centering
  \caption{「た」活用示例}
  \begin{tabular}{c c c c c c c}
    基本形 & 未然形 & 连用形 & 终止形 & 连体形 & 假定形 & 命令形 \\
    \hline
    た & たろ & & た & た & たら & \\
  \end{tabular}
\end{table}


\subsection{构成敬体的助动词}%

\subsubsection{ます}%

接在动词和动词型助动词的连用形之后,
表示说话者和作者的一种郑重的心情,
表示对听者和读者的尊敬,
构成敬体。
广泛用于谈话中。

「ます」属于特殊活用。

\begin{table}[h]
  \centering
  \caption{「ます」活用示例}
  \begin{tabular}{c c c c c c c}
    基本形 & 未然形 & 连用形 & 终止形 & 连体形 & 假定形 & 命令形 \\
    \hline
    ます & \makecell{ませ \\ ましょ} & まし & ます & ます & ますれ & \makecell{ませ \\ まし} \\
  \end{tabular}
\end{table}

\begin{itemize}
  \item 私わ お茶を飲みません。
  \item みんなもう帰りました。
  \item 毎日、六時に起きます。
  \item いらっしゃいませ。
\end{itemize}


\subsubsection{です}%

接在形容词、形容词型助动词之后,
表示说话者和作者的一种郑重的心情,
表示对听者和读者的尊敬,
构成敬体。

「ます」属于特殊活用,一般只使用其未然形和终止形。

\begin{table}[h]
  \centering
  \caption{「です」活用示例}
  \begin{tabular}{c c c c c c c}
    基本形 & 未然形 & 终止形 \\
    \hline
    です & でしょ & です \\
  \end{tabular}
\end{table}



\subsection{表示各种陈述方式的助动词}%

\subsubsection{たい、たがる}%

接在动词的连用形之后,
表示说话人内心的愿望,
相当于汉语``想······''。
「たい」表示说话人内心的愿望,通常用于第一人称。
「たがる」表示显露与言行的愿望,通常用于第三人称。

「たい」属于形容词型活用,
「たがる」属于五段动词型活用。

\begin{table}[h]
  \centering
  \caption{「たい」活用示例}
  \begin{tabular}{c | c | c c c c c}
    基本形 & 未然形 & 连用形 & 终止形 & 连体形 & 假定形 \\
    \hline
    たい & たかろ & \makecell{\cn[1] たく \\ \cn[2] たかっ} & たい & たい & たけれ \\
    たがる & \makecell{\cn[1] たがら \\ \cn[2] たがろ} & \makecell{\cn[1] たがり \\ \cn[2] たがっ} & たがる & たがる & たがれ \\
  \end{tabular}
\end{table}

\begin{itemize}
  \item 早く帰りたかろう。
  \item 私は 薬を飲みたくないです。
  \item 昨日、家へ帰りたかったです。
  \item 子供はお菓子を食べだかっています。
\end{itemize}


\subsubsection{ない}%

接在动词的未然形之后,
表示对动作、作用、状态、属性的否定,
相当于汉语``不······''。

「ない」属于形容词型活用。

\begin{table}[h]
  \centering
  \caption{「たい」活用示例}
  \begin{tabular}{c | c | c c c c c}
    基本形 & 未然形 & 连用形 & 终止形 & 连体形 & 假定形 \\
    \hline
    ない & なかろ & \makecell{\cn[1] なく \\ \cn[2] なかっ} & ない & ない & なけれ \\
  \end{tabular}
\end{table}

\begin{itemize}
  \item 私はそのことを何も知らない。
\end{itemize}

\subsubsection{である}

断定助动词,接在体言后,表示对事物的断定,属于文章语。
词尾变化同五段动词。
否定式为「ではない」。
\begin{itemize}
  \item 鯨は哺乳動物であり、魚類ではない。
\end{itemize}


\subsubsection{ようだ}%

比况助动词「ようだ」,
接在用言连体形、「体言の」之后,
表示比喻、示例以及根据莫种情况进行的判断。
\begin{itemize}
  \item 表示比喻:まるで夢のようだ。
  \item 表示示例:サッカーやラグビーのような激しい運動が好きだ。
  \item 表示判断:図書館にはだれもいないようです。
\end{itemize}

「ようだ」属于形容动词型活用,敬体是「ようです」。
\begin{table}[h]
  \centering
  \caption{「ようだ」活用示例}
  \begin{tabular}{c | c | c c c c c}
    基本形 & 未然形 & 连用形 & 终止形 & 连体形 & 假定形 \\
    \hline
    ようだ & ようだろ & \makecell{\cn[1] ようだっ\\ \cn[2] ようで} & ようだ & ような & ようなら \\
    ようです & ようでしょ & ようでし & ようです & &  \\
  \end{tabular}
\end{table}


\subsubsection{そうだ}%

样态助动词「そうだ」,
接在动词连用形、形容词词干、形容动词词干后,
表示根据事物表现的样子、迹象,
推测判断某种情况即将出现或某事物可能具有某种性质。
\begin{itemize}
  \item このりんごはおいしそうだ。
  \item 今にも、雨が降りそうだ。
  \item この問題は簡単そうで、難しいです。
\end{itemize}

「そうだ」属于形容动词型活用,敬体是「そうです」。
\begin{table}[h]
  \centering
  \caption{「そうだ」活用示例}
  \begin{tabular}{c | c | c c c c c}
    基本形 & 未然形 & 连用形 & 终止形 & 连体形 & 假定形 \\
    \hline
    そうだ & そうだろ & \makecell{\cn[1] そうだっ\\ \cn[2] そうで} & そうだ & そうな & そうなら \\
    そうです & そうでしょ & そうでし & そうです & &  \\
  \end{tabular}
\end{table}

