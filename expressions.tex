\section{常用词汇和表达}%
\label{sec:exp}

{\bf
\noindent 动词连用形+ています
}

接在持续性动词后,表示正在进行的动作或持续的状态。
\begin{itemize}
  \item 空港の到着ロビーで待っています。
\end{itemize}

{\bf
\noindent 动词连用形+てください
}

表示请求。
\begin{itemize}
  \item 書類にお名前を書いてください。
\end{itemize}

{\bf
\noindent 动词连用形+てから
}

表示一个动作完成后,再进行另一个动作。
\begin{itemize}
  \item 水泳をしてから、学校へ行きます。
\end{itemize}

{\bf
\noindent 体言+について
}

表示陈述的内容,相当于汉语``关于······''。
\begin{itemize}
  \item この問題について検討しましょう。
\end{itemize}

{\bf
\noindent 決して+用言否定式
}

表示强烈否定,相当于汉语``绝不······''。
\begin{itemize}
  \item 決して許しません。
\end{itemize}

{\bf
\noindent 体言+は言うまでもない
}

相当于汉语``不用说······''。
\begin{itemize}
  \item 英語は言うまでもなく、日本語もできます。
\end{itemize}

{\bf
\noindent 体言+という+体言
}

构成同位语。
\begin{itemize}
  \item 電車に乗ると、「携帯電話はご遠慮ください」という放送が聞こえます。
\end{itemize}

{\bf
\noindent 体言の、用言连体形+ほうがいいです
}

多用于建议对方采取某种行为,相当于汉语``还是······为好''。
\begin{itemize}
  \item 明日は早く出発するから、早く寝たほうがいいです。
\end{itemize}

{\bf
\noindent 体言、用言终止形+かもしれません
}

表示对事物的估计,相当于汉语``也许······''。
\begin{itemize}
  \item 雨が降るかもしれません。
\end{itemize}

{\bf
\noindent 体言、用言终止形+に違いありません
}

表示确信,相当于汉语``一定是······''。
\begin{itemize}
  \item 彼はきっと成功するに違いありません。
\end{itemize}

{\bf
\noindent 动词未然形+なければなりません
}

相当于汉语``必须······''。
\begin{itemize}
  \item 今週中に宿題を出さなければなりません。
  \item もう8時ですから、学校へ行かなければなりません。
\end{itemize}

{\bf
\noindent 动词、形容词连用形+てもいいです
}

表示许可,相当于汉语``可以······''。
\begin{itemize}
  \item この小説を借りてもいいですか。
  \item 明日は休みですから、遅くまで起きていてもいいです。
\end{itemize}

{\bf
\noindent 动词连用形+てはいけません
}

表示禁止,相当于汉语``不许······''。
\begin{itemize}
  \item 危ないから、そんなことをしてはいけません。
  \item 先生にそんなことを言ってはいけません。
\end{itemize}

{\bf
\noindent 动词未然形+ないでください
}

表示禁止,相当于汉语``请不要······''。
\begin{itemize}
  \item ここでタバコを吸わないでください。
  \item 授業の時間ですから、廊下で騒がないでください。
\end{itemize}

{\bf
\noindent 动词连用形+てくる
}

表示时间或空间的发展或趋势,相当于汉语``······起来了''。
\begin{itemize}
  \item われわれの生活は日に日によくなってきた。
\end{itemize}

{\bf
\noindent 动词连用形+なさい
}

表示不客气的请求或命令,相当于汉语``你······吧''。
\begin{itemize}
  \item 早く帰りなさい。
\end{itemize}

{\bf
\noindent 体言+によって
}

相当于汉语``因······而异''。
\begin{itemize}
  \item 先生によって、教え方も違う。
\end{itemize}

{\bf
\noindent 体言+に対して
}

相当于汉语``对于······''。
\begin{itemize}
  \item その決定に対して抗議した。
\end{itemize}

{\bf
\noindent 动词连用形+「てはならない」
}

表示道理上不允许,相当于汉语``不可以······''。
\begin{itemize}
  \item 忙しくても、親に手紙を書くのを忘れてはならない。
\end{itemize}

{\bf
\noindent 体言+「とは」
}

提示主题,相当于汉语``所谓······''。
\begin{itemize}
  \item 週刊誌とは、毎週一回出る雑誌のことです。
\end{itemize}

{\bf
\noindent 动词连用形+「ていく」
}

表示动作、状态的发展趋势,相当于汉语``······起来''。
\begin{itemize}
  \item 少しずつ慣れていく。
\end{itemize}

{\bf
\noindent 动词连体形+「に従って」
}

相当于汉语``随着······''。
\begin{itemize}
  \item 新しい技術の開発が進歩に従って、生産規模も拡大した。
\end{itemize}

{\bf
\noindent 动词连体形、体言+「につれて」
}

相当于汉语``随着······''。
\begin{itemize}
  \item 寮生活に慣れるにつれて、よく眠れるようになった。
\end{itemize}

{\bf
\noindent 动词连用形+「てしまう」
}

表示动作的完成。
\begin{itemize}
  \item 作文はもう書いてしまいました。
\end{itemize}

{\bf
\noindent 动词连体形+「ところだ」
}

表示将要进行某一动作,相当于汉语``正要······''。
\begin{itemize}
  \item 今、出掛けるとこです。
\end{itemize}

{\bf
\noindent 动词连用形+「ている」+「ところだ」
}

表示正在进行某一动作,相当于汉语``正在······''。
\begin{itemize}
  \item みんなは今、食事をしているところです。
\end{itemize}

{\bf
\noindent 动词连用形+「た」+「ところだ」
}

表示刚刚完成某一动作,相当于汉语``刚刚······''。
\begin{itemize}
  \item 飛行機は今、飛び立ったところです。
\end{itemize}

{\bf
\noindent 动词连用形+「てみる」
}

表示尝试做某事,相当于汉语``试着······''。
\begin{itemize}
  \item 自分で調べてみてください。
\end{itemize}

{\bf
\noindent 用言终止形、体言+「とはいえ」
}

相当于汉语``虽然······,但是''。
\begin{itemize}
  \item 春とはいえ、まだ風が冷たい。
\end{itemize}

\begin{table}[h]
  \centering
  \caption{常用表达}
  \label{tab:label}
  \small
  \begin{tabular}{ll}
    初めましで & 初次见面 \\
    よろしく お願いします & 请多关照 \\
    今日は & 你好 \\
    すみません & 对不起 \\
    御免なさい & 对不起 \\
    おはようございます & 早上好 \\
    お出掛けですか & 您出门啊 \\
    行って参ります & 我走了 \\
    おめでとうございます & 恭喜 \\
    お誕生日おめでとうございます & 祝你生日快乐 \\
    行っていらっしゃる & 走好 \\
  \end{tabular}
\end{table}

\begin{table}[h]
  \centering
  \caption{常用指代词汇}
  \label{tab:label}
  \small
  \begin{tabular}{c|ccc|cc|c}
    & \multicolumn{3}{c|}{指示代词} & \multicolumn{2}{c|}{连体词} & 副词 \\
    & 事物 & 场所 & 方向 & 事物 & 性质状态 & 状态 \\
    \hline
    近 & これ & ここ   & こちら & この & こんな & こんなに \\
    中 & それ & そこ   & そちら & その & そんな & そんなに \\
    远 & あれ & あそこ & あちら & あの & あんな & あんなに \\
    ?  & どれ & どこ   & どちら & どの & どんな & どんなに \\
  \end{tabular}
\end{table}

\begin{table}[h]
  \centering
  \caption{人称代名詞}
  \label{tab:label}
  \small
  \begin{tabular}{c|c|c|ccc|c}
    & \multirow{2}{*}{第一人称} & \multirow{2}{*}{第二人称} & \multicolumn{3}{c|}{第三人称} & \multirow{2}{*}{不定称} \\
    & & & 近 & 中 & 远 & \\
    \hline
    单数 & わたし & あなた & このひと & そのひと & あのひと & どのひと \\
    复数 &わたしたち & あなたたち & このひとたち & そのひとたち & あのひとたち & どのひとたち \\
  \end{tabular}
\end{table}

\begin{table}[h]
  \centering
  \caption{常用数词}
  \label{tab:number}
  \small
  \begin{tabular}{ll | ll | ll}
    一 & いち \cn[2]                  & 百   &  ひゃく \cn[2]     & 一つ &  ひとつ \cn[2] \\
    二 & に \cn[1]                    & 三百 &  さんびゃく \cn[1] & 二つ &  ふたつ \cn[3] \\
    三 & さん \cn[0]                  & 六百 &  ろっぴゃく \cn[4] & 三つ &  みっつ \cn[3] \\
    四 & し \cn[1] $\,$ よん \cn[1]   & 八百 &  はっぴゃく \cn[4] & 四つ &  よっつ \cn[3] \\
    五 & ご \cn[1]                    & 千   &  せん \cn[1]       & 五つ &  いつつ \cn[2] \\
    六 & ろく \cn[2]                  & 三千 &  さんぜん \cn[3]   & 六つ &  むっつ \cn[3] \\
    七 & しち \cn[2] $\,$ なな \cn[1] & 八千 &  はっせん \cn[3]   & 七つ &  ななつ \cn[2] \\
    八 & はち \cn[2]                  & 万   &  まん \cn[1]       & 八つ &  やっつ \cn[3] \\
    九 & く \cn[1] $\,$ きゅう \cn[1] & 億   &  おく \cn[1]       & 九つ &  ここのつ \cn[2] \\
    十 & じゅう \cn[1]                &      &                    & 十つ &  とお \cn[1] \\
  \end{tabular}
\end{table}

\begin{table}[h]
  \centering
  \caption{常用量词}
  \label{tab:number}
  \small
  \begin{tabular}{lll | ll}
    人 & にん & 人           & 何人 & なんにん  \\
    時 & じ   & 点           & 何時 & なんじ    \\
    分 & ふん & 分           & 何分 & なんふん  \\
    番 & ばん & 第~         & 何番 & なんばん  \\
    枚 & まい & 张(扁平物) & 何枚 & なんまい  \\
    回 & かい & 回,次       & 何回 & なんかい  \\
    歳 & さい & 岁           & 何歳 & なんさい  \\
    冊 & さつ & 册           & 何冊 & なんさつ  \\
    本 & ほん & 根(细长物) & 何本 & なんほん  \\
    台 & だい & 台,辆       & 何台 & なんだい  \\
    足 & そく & 双           & 何足 & なんそく  \\
  \end{tabular}
\end{table}

\begin{table}[h]
  \centering
  \caption{日期}
  \label{tab:number}
  \small
  %\scriptsize
  \begin{tabular}{ll | ll | ll}
    一月   & いちがつ \cn[4]       & 一日     & ついたち \cn[4] & 日曜日 & にちようび \cn[3] \\
    二月   & にがつ \cn[3]         & 二日     & ふつか \cn[0]   & 月曜日 & げつうび   \cn[3] \\
    三月   & さんがつ \cn[1]       & 三日     & みっか \cn[0]   & 火曜日 & かようび   \cn[2] \\
    四月   & しがつ \cn[3]         & 四日     & よっか \cn[0]   & 水曜日 & すいようび \cn[3] \\
    五月   & ごがつ \cn[1]         & 五日     & いつか \cn[0]   & 木曜日 & もくようび \cn[3] \\
    六月   & ろくがつ \cn[0]       & 六日     & むいか \cn[0]   & 金曜日 & きんようび \cn[3] \\
    七月   & しちがつ \cn[0]       & 七日     & なのか \cn[0]   & 土曜日 & どようび   \cn[2] \\
    八月   & はちがつ \cn[4]       & 八日     & ようか \cn[0]   & 何曜日 & なんようび \cn[3] \\
    九月   & くがつ \cn[1]         & 九日     & ここのか \cn[0] &        & \\
    十月   & じゅうがつ \cn[4]     & 十日     & とおか \cn[0]   &        & \\
    十一月 & じゅういちがつ \cn[6] & 十四日   & じゅうよっか \cn[1] +\cn[0] &        & \\
    十二月 & じゅうにがつ \cn[5]   & 二十日   & はつか \cn[0]   &        & \\
    何月   & なんがつ \cn[1]       & 二十四日 & にじゅうよっか \cn[1] +\cn[0] &        & \\
    何年   & なんがつ \cn[1]       & 何日     & なんにち \cn[1] &        & \\
  \end{tabular}
\end{table}

\begin{table}[h]
  \centering
  \caption{地名}
  \label{tab:number}
  \small
  \begin{tabular}{lll | lll}
    名古屋 & なごや \cn[1]     & Nagoya  & 東京 & とうきょう \cn[0] & Tokyo \\
    仙台   & せんだい \cn[1]   & Sendai  & 沖縄 & おきなわ \cn[0]   & Okinawa \\
    新宿   & しんじゅく \cn[0] & Sinjuku & 原宿 & はらじゅく \cn[2] & Harajuku \\
  \end{tabular}
\end{table}

\begin{table}[h]
  \centering
  \caption{人名后缀}
  \label{tab:number}
  \small
  \begin{tabularx}{\textwidth}{llX}
    ちゃん & & 关系最为亲密的一种称呼,是「さん」的转音,接在名字后表示亲热。可以用在关系较好、彼此比较熟悉的朋友或夫妻、家人之间。 \\
    君 & くん & 称呼朋友或是年龄、资历比自己低的后辈时使用。常用于称呼男同学、男下属。
                带有亲密感,与「ちゃん」相比还略带一些敬意在里边,不熟悉的人之间一般不用。\\
    さん & & 相当于汉语中的 ``女士''、``先生''、``同志''、``同学'',日语语感中这个称呼既带有敬意也有亲密感在里边。\\
    様 & さま & 比「さん」更为敬重的一种表达,相当于汉语中的``大人''。多用于称呼比自己年长、地位较高的人。\\
    殿 & どの & 极为尊敬的一种用法,但是这种用法口语中不会使用,多用于奖状或毕业证等正式文书中。\\
    殿下 & でんか & 就是 ``殿下'' 的意思,是对皇太子、皇太子妃、皇太孙等皇室亲族的称呼。 \\
    陛下 & へいか & 对天皇以及皇后、太皇太后、皇太后的尊称。\\
  \end{tabularx}
\end{table}


