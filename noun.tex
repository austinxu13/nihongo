\section{名詞}%

\subsection{形式名詞}%

形式上是名词,但却没有或很少有实质意义、
只能和修饰它的定语一起使用而不能独立使用的一类名词。

{\bf
\noindent つもり
}

表示计划或打算。
\begin{itemize}
  \item 私は日本へ行くつもりです。
\end{itemize}

{\bf
\noindent こと
}

泛指事情。
\begin{itemize}
  \item 冬は雪が降ることが多いです。
\end{itemize}

动词连体形+ことがあります,表示有时发生:
\begin{itemize}
  \item 私は東京へ出張に行くことがあります。
\end{itemize}

动词过去式+ことがあります,表示曾发生过:
\begin{itemize}
  \item 私は刺身お食べたことがあります。
\end{itemize}

{\bf
\noindent の
}

泛指人、事、物。
\begin{itemize}
  \item 留学が簡単なのはいいことです。
\end{itemize}

{\bf
\noindent ため
}

表示前项是后项的目标。
\begin{itemize}
  \item 人は食べるために生きるのではなく、生きるために食べるのです。
\end{itemize}

{\bf
\noindent はず
}

表示有把握的判断。
\begin{itemize}
  \item 電車は5時に来るはずだ。
  \item 彼にしらせたから、知っているはずだ。
\end{itemize}


\subsection{常见句型}%

{\bf
\noindent 体言+について
}

表示陈述的内容,相当于汉语``关于······''。
\begin{itemize}
  \item この問題について検討しましょう。
\end{itemize}


{\bf
\noindent 体言+は言うまでもない
}

相当于汉语``不用说······''。
\begin{itemize}
  \item 英語は言うまでもなく、日本語もできます。
\end{itemize}

{\bf
\noindent 体言+という+体言
}

构成同位语。
\begin{itemize}
  \item 電車に乗ると、「携帯電話はご遠慮ください」という放送が聞こえます。
\end{itemize}

{\bf
\noindent 体言+によって
}

相当于汉语``因······而异''。
\begin{itemize}
  \item 先生によって、教え方も違う。
\end{itemize}

{\bf
\noindent 体言+に対して
}

相当于汉语``对于······''。
\begin{itemize}
  \item その決定に対して抗議した。
\end{itemize}

{\bf
\noindent 体言+「とは」
}

提示主题,相当于汉语``所谓······''。
\begin{itemize}
  \item 週刊誌とは、毎週一回出る雑誌のことです。
\end{itemize}


