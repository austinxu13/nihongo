\section{名詞}%

\subsection{形式名詞}%

形式上是名词,但却没有或很少有实质意义、
只能和修饰它的定语一起使用而不能独立使用的一类名词。

{\bf
\noindent つもり
}

表示计划或打算。
\begin{itemize}
  \item 私は日本へ行くつもりです。
\end{itemize}

{\bf
\noindent こと
}

泛指事情。
\begin{itemize}
  \item 冬は雪が降ることが多いです。
\end{itemize}

动词连体形+ことがあります,表示有时发生:
\begin{itemize}
  \item 私は東京へ出張に行くことがあります。
\end{itemize}

动词过去式+ことがあります,表示曾发生过:
\begin{itemize}
  \item 私は刺身お食べたことがあります。
\end{itemize}

{\bf
\noindent の
}

泛指人、事、物。
\begin{itemize}
  \item 留学が簡単なのはいいことです。
\end{itemize}

{\bf
\noindent ため
}

表示前项是后项的目标。
\begin{itemize}
  \item 人は食べるために生きるのではなく、生きるために食べるのです。
\end{itemize}
