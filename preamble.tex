% This file contains all the default packages and modifications
% for LHCb formatting

%% %%%%%%%%%%%%%%%%%%
%%  Page formatting
%% %%%%%%%%%%%%%%%%%%
%%\usepackage[margin=1in]{geometry}
\usepackage[top=1.5in, bottom=1.5in, left=1in, right=1in]{geometry}
%\usepackage[top=1.5in, bottom=2in, left=1.35in, right=1.35in]{geometry}

% fallback for manual settings... uncomment if the geometry package is not available
%\voffset=-11mm
%\textheight=220mm
%\textwidth=160mm
%\oddsidemargin=0mm
%\evensidemargin=0mm

\columnsep=5mm
\addtolength{\belowcaptionskip}{0.5em}

\renewcommand{\textfraction}{0.01}
\renewcommand{\floatpagefraction}{0.8} % changed from 0.99
\renewcommand{\topfraction}{0.9}
\renewcommand{\bottomfraction}{0.9}

\renewcommand{\baselinestretch}{1.5}

%\usepackage{setspace}
%\setstretch{1.50}

% Allow the page size to vary a bit
\raggedbottom
% To avoid Latex to be too fussy with line breaking
\sloppy

%% %%%%%%%%%%%%%%%%%%%%%
%% Packages to be used
%% %%%%%%%%%%%%%%%%%%%%% 
\usepackage{microtype}
\usepackage{lineno}  % for line numbering during review
\usepackage{xspace} % To avoid problems with missing or double spaces after predefined symbold
\usepackage{caption} %these three command get the figure and table captions automatically small
\renewcommand{\captionfont}{\small}
\renewcommand{\captionlabelfont}{\small}

%% Graphics
\usepackage{graphicx}  % to include figures (can also use other packages)
\usepackage{color}
\usepackage{colortbl}
\graphicspath{{./figs/}} % Make Latex search fig subdir for figures
\DeclareGraphicsExtensions{.pdf,.PDF,png,.PNG}

%% Math
\usepackage{amsmath} % Adds a large collection of math symbols
\usepackage{amssymb}
\usepackage{amsfonts}
\usepackage{upgreek} % Adds in support for greek letters in roman typeset

%% fix to allow peaceful coexistence of line numbering and
%% mathematical objects
%% http://www.latex-community.org/forum/viewtopic.php?f=5&t=163
\newcommand*\patchAmsMathEnvironmentForLineno[1]{%
\expandafter\let\csname old#1\expandafter\endcsname\csname #1\endcsname
\expandafter\let\csname oldend#1\expandafter\endcsname\csname
end#1\endcsname
 \renewenvironment{#1}%
   {\linenomath\csname old#1\endcsname}%
   {\csname oldend#1\endcsname\endlinenomath}%
}
\newcommand*\patchBothAmsMathEnvironmentsForLineno[1]{%
  \patchAmsMathEnvironmentForLineno{#1}%
  \patchAmsMathEnvironmentForLineno{#1*}%
}
\AtBeginDocument{%
\patchBothAmsMathEnvironmentsForLineno{equation}%
\patchBothAmsMathEnvironmentsForLineno{align}%
\patchBothAmsMathEnvironmentsForLineno{flalign}%
\patchBothAmsMathEnvironmentsForLineno{alignat}%
\patchBothAmsMathEnvironmentsForLineno{gather}%
\patchBothAmsMathEnvironmentsForLineno{multline}%
\patchBothAmsMathEnvironmentsForLineno{eqnarray}%
}

% Get hyperlinks to captions and in references.
% These do not work with revtex. Use "hypertext" as class option instead.
\usepackage{hyperref}
\usepackage[all]{hypcap} % Internal hyperlinks to floats.

% overleaf comments
\usepackage[colorinlistoftodos,textsize=scriptsize]{todonotes}


% for number display
\usepackage{siunitx}
\sisetup{separate-uncertainty}

% For table format
\usepackage{makecell}

% multiple reference
\usepackage{cleveref}

% rotate tables
\usepackage{rotating}

\usepackage{longtable} % only for template; not usually to be used in PAPERs

% Chinese
\usepackage{fontspec}
\usepackage{xeCJK}
\usepackage{xpinyin}
\usepackage{ruby}
\renewcommand{\rubysize}{0.75}
\setCJKmainfont{STSong}
\setCJKmainfont[BoldFont={STHeiti}, ItalicFont={STKaiti}]{STSong}
%  \setCJKmainfont[BoldFont={STXihei}, ItalicFont={STXingkai}]{STSong}
%  \setCJKsansfont{STXihei}
%  \setCJKmonofont{STXingkai}
\setCJKfamilyfont{song}{STSong}
\setCJKfamilyfont{hei}{STHeiti}
\setCJKfamilyfont{fs}{STFangsong}
%\setCJKfamilyfont{kai}{STXingkai}
%\setCJKfamilyfont{li}{STLiti} % todo: 用隶书字体代替
%\setCJKfamilyfont{you}{Yuanti SC} % todo: 用幼圆字体代替
%\setmainfont{Times New Roman}
%\setsansfont{Arial}
%\setmonofont{Courier New}
\newcommand{\song}{\CJKfamily{song}}    % 宋体
\def\songti{\song}
\newcommand{\fsong}{\CJKfamily{fs}}     % 仿宋体
\def\fangsong{\fsong}
\newcommand{\kai}{\CJKfamily{kai}}      % 楷体
\def\kaishu{\kai}
\newcommand{\hei}{\CJKfamily{hei}}      % 黑体
\def\heiti{\hei}
\newcommand{\li}{\CJKfamily{li}}        % 隶书
\def\lishu{\li}
\newcommand{\you}{\CJKfamily{you}}      % 幼圆
\def\youyuan{\you}
\newcommand{\xiaosi}{\fontsize{12bp}{14.4bp}\selectfont}


\newtheorem{definition}{定义}[section]
\newtheorem{theorem}{定理}[section]
\newtheorem{proposition}{命题}[section]
\newtheorem{lemma}{引理}[section]
\renewcommand\abstractname{摘要}
\renewcommand\contentsname{目录}
\renewcommand\figurename{图}
\renewcommand\tablename{表}

\usepackage{extarrows}

\usepackage{enumerate}

\usepackage{booktabs}

\usepackage{multirow}

\def \cn[#1] {\raisebox{.5pt}{\textcircled{\raisebox{-.9pt} {#1}}}}

\usepackage{supertabular}

\usepackage{tabularx}

\usepackage{changepage}

