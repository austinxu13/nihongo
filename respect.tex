\section{敬語}%

敬语是一种语言待遇。
敬语的使用涉及两个方面的对象,
一是说话中提到的人物事项,
一是听话人。
对于这两方面的对象,
都可以从上下关系、亲疏关系、场合与口吻等三种情况考虑。

敬语分为四类:
\begin{itemize}
  \item 尊敬语。
  \item 谦让语。
  \item 郑重语:说话人对听话人表示敬意的一种语言表达方式,
    主要靠出现在句末的助动词「です」、「ます」表示。
    郑重语就是敬体。
    以用言或助动词的终止形结句的文体叫做简体。
  \item 美化语。
\end{itemize}


\subsection{体言的尊敬语}%

\subsubsection{本身是尊敬语的体言}%

\subsubsection{通过前后缀构成尊敬语}%

\begin{itemize}
  \item 前缀「お」:お元気、お世話、お礼、お茶
  \item 前缀「ご」:ご感想、ご指導
  \item 后缀「様」:いろいろお世話様でした。
  \item 后缀「さん」
\end{itemize}

\subsection{动词的尊敬语}%

\subsubsection{尊敬动词}%

若干常见的动词有对应形式的尊敬动词:
\begin{itemize}
  \item いらっしゃる:行く、来る、いる
  \item おっしゃる:言う
  \item たさる:する
  \item くださる:くれる
\end{itemize}

\subsubsection{将一般动词变成尊敬语形式}%

\begin{itemize}
  \item お+五段动词、一段动词连用形+になる
  \item ご+サ变动词词干+になる
  \item お+五段动词、一段动词连用形+なさる
  \item ご+サ变动词词干+なさる
  \item お+五段动词、一段动词连用形+くださる
  \item ご+サ变动词词干+くださる
\end{itemize}
