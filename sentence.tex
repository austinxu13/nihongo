\section{文}%
\label{sec:_}

%%%%%%%%%%%%%%%%%%%%
\subsection{断定文}%
%%%%%%%%%%%%%%%%%%%%

{\bf
\noindent 体言は 体言です

\noindent 体言は 体言ではありません
}

\begin{enumerate}
  \item 私わ 大学院生です。
  \item 彼女わ 主婦では ありません。
  \item 田中さんは 先生でしょう。
  \item 今日は 水曜日で、昨日は 火曜日でした。
  \item 昨日は 休みでは ありませんでした。
\end{enumerate}


%%%%%%%%%%%%%%%%%%%%
\subsection{描写文}%
%%%%%%%%%%%%%%%%%%%%

{\bf
\noindent 体言は 形容詞です
}

\begin{enumerate}
  \item 東京の夏は 蒸し暑いです。
\end{enumerate}

{\bf
\noindent 体言は 形容動詞
\noindent 体言は 形容動詞词干です
}

\begin{enumerate}
  \item 街は 清潔です。
\end{enumerate}


%%%%%%%%%%%%%%%%%%%%
\subsection{叙述文}%
%%%%%%%%%%%%%%%%%%%%

{\bf
\noindent 体言は 動詞
}

\begin{enumerate}
  \item 私は 六時に 起きます。
  \item 教室で 予習を します。
\end{enumerate}


%%%%%%%%%%%%%%%%%%%%
\subsection{存在文}%
%%%%%%%%%%%%%%%%%%%%

{\bf
\noindent 体言は 体言にあります

\noindent 体言は 体言にいます
}

\begin{enumerate}
  \item 兄は 名古屋に います。
  \item 花瓶は 机の上に あります。
  \item 実験室は どこに ありますか。
\end{enumerate}

{\bf
\noindent 体言には 体言があります

\noindent 体言には 体言がいます
}

\begin{enumerate}
  \item 学校には 図書館が あります。
  \item 部屋の中には 猫が いますか?
\end{enumerate}

