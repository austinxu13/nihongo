\section{動詞}%


%%%%%%%%%%%%%%%%%%%%%%%%
\subsection{动词的分类}%
%%%%%%%%%%%%%%%%%%%%%%%%

\noindent 根据活用规律:五段动词、一段动词、サ变动词、カ变动词。

\noindent 根据是否要求宾语:他动词、自动词。

\noindent 根据后接「ている」的情况
\begin{itemize}
  \item 继续动词后接「ている」,表示动作正在进行:雁が大空を渡っている。
  \item 瞬间动词后接「ている」,表示动作已经结束但结果还保留着:小説発行の準備が始まっている。
  \item 形容词性动词后接「ている」,表示状态、性质。
\end{itemize}



%%%%%%%%%%%%%%%%%%%%%%%%%%
\subsection{动词的活用形}%
%%%%%%%%%%%%%%%%%%%%%%%%%%

动词的活用形见表~\ref{tab:verb}.

\begin{table}[htb]
  \centering
  \caption{动词活用示例}
  \label{tab:verb}
  \scriptsize
  \begin{tabular}{c | c | c | c c c c c c}
    分类 & 词例 & 词干 & 未然形 & 连用形 & 终止形 & 连体形 & 假定形 & 命令形 \\
    \hline
    \multirow{9}{*}{五段}
    & 書く & \ruby{書}{か}   & \makecell{\cn[1] か \\ \cn[2] こ} & き & く & く & け & け \\
    & 泳ぐ & \ruby{泳}{およ} & \makecell{\cn[1] が \\ \cn[2] ご} & ぎ & ぐ & ぐ & げ & げ \\
    & 話す & \ruby{話}{はな} & \makecell{\cn[1] さ \\ \cn[2] そ} & し & す & す & せ & せ \\
    & 立つ & \ruby{立}{た}   & \makecell{\cn[1] た \\ \cn[2] と} & ち & つ & つ & て & て \\
    & 取る & \ruby{取}{と}   & \makecell{\cn[1] ら \\ \cn[2] ろ} & り & る & る & れ & れ \\
    & 歌う & \ruby{歌}{うた} & \makecell{\cn[1] わ \\ \cn[2] お} & い & う & う & え & え \\
    & 死ぬ & \ruby{死}{し}   & \makecell{\cn[1] な \\ \cn[2] の} & に & ぬ & ぬ & ね & ね \\
    & 飛ぶ & \ruby{飛}{と}   & \makecell{\cn[1] ば \\ \cn[2] ぼ} & び & ぶ & ぶ & べ & べ \\
    & 読む & \ruby{読}{よ}   & \makecell{\cn[1] ま \\ \cn[2] も} & み & む & む & め & め \\
    \hline
    \multirow{2}{*}{上一段}
    & 起きる & \ruby{起}{お} & き & き & きる & きる & きれ & \makecell{きろ\\きよ} \\
    & 見る & \ruby{見}{み}   & み & み & みる & みる & みれ & \makecell{みろ\\みよ} \\
    \hline
    \multirow{2}{*}{下一段}
    & 食べる & \ruby{食}{た} & べ & べ & べる & べる & べれ & \makecell{べろ\\べよ} \\
    & 寝る & \ruby{寝}{ね}   & ね & ね & ねる & ねる & ねれ & \makecell{ねろ\\ねよ} \\
    \hline
    \multirow{2}{*}{サ变动词}
    & する &  & \makecell{し\\せ} & し & する & する & すれ & \makecell{しよ\\せよ} \\
    & 勉強する & \ruby{勉強}{べんきょう} & \makecell{し\\せ} & し & する & する & すれ & \makecell{しよ\\せよ} \\
    \hline
    カ变动词 & 来る & & \ruby{来}{こ} & \ruby{来}{き} & \ruby{来}{く}る & \ruby{来}{く}る & \ruby{来}{く}れ & \ruby{来}{こ}い \\
  \end{tabular}
\end{table}


\subsubsection{未然形}%
\label{ssub:_}

不能独立出现,必须后续助动词,才可表达完整的语法意义。
未然形后主要接一下几种助动词:
\begin{itemize}
  \item \cn[1] +「ない」,表示否定
  \item \cn[1] +「られる」等,构成不同的态
  \item \cn[2] +「う」、「よう」等,表示意志:私がご飯を作ろう。
\end{itemize}

{\bf
\noindent 決して+用言否定式
}

表示强烈否定,相当于汉语``绝不······''。
\begin{itemize}
  \item 決して許しません。
\end{itemize}

{\bf
\noindent 动词未然形+なければなりません
}

相当于汉语``必须······''。
\begin{itemize}
  \item 今週中に宿題を出さなければなりません。
  \item もう8時ですから、学校へ行かなければなりません。
\end{itemize}

{\bf
\noindent 动词未然形+ないでください
}

表示禁止,相当于汉语``请不要······''。
\begin{itemize}
  \item ここでタバコを吸わないでください。
  \item 授業の時間ですから、廊下で騒がないでください。
\end{itemize}


\subsubsection{连用形}%

\begin{enumerate}
  \item 充当名词或其他词素,构成复合词。
    \begin{itemize}
      \item 蒸し暑い
    \end{itemize}
  \item 表示中顿。
  \item 后接助动词「ます」、「た」、「たい」、「そうだ」等。
  \item 后接助词「に」、「て」、「たり」等。
\end{enumerate}

{\bf
\noindent 动词连用形+ています
}

接在持续性动词后,表示正在进行的动作或持续的状态。
\begin{itemize}
  \item 空港の到着ロビーで待っています。
\end{itemize}

{\bf
\noindent 动词连用形+てください
}

表示请求。
\begin{itemize}
  \item 書類にお名前を書いてください。
\end{itemize}

{\bf
\noindent 动词连用形+てから
}

表示一个动作完成后,再进行另一个动作。
\begin{itemize}
  \item 水泳をしてから、学校へ行きます。
\end{itemize}

{\bf
\noindent 动词、形容词连用形+てもいいです
}

表示许可,相当于汉语``可以······''。
\begin{itemize}
  \item この小説を借りてもいいですか。
  \item 明日は休みですから、遅くまで起きていてもいいです。
\end{itemize}

{\bf
\noindent 动词连用形+てはいけません
}

表示禁止,相当于汉语``不许······''。
\begin{itemize}
  \item 危ないから、そんなことをしてはいけません。
  \item 先生にそんなことを言ってはいけません。
\end{itemize}

{\bf
\noindent 动词连用形+てくる
}

表示时间或空间的发展或趋势,相当于汉语``······起来了''。
\begin{itemize}
  \item われわれの生活は日に日によくなってきた。
\end{itemize}

{\bf
\noindent 动词连用形+なさい
}

表示不客气的请求或命令,相当于汉语``你······吧''。
\begin{itemize}
  \item 早く帰りなさい。
\end{itemize}

{\bf
\noindent 动词连用形+「てはならない」
}

表示道理上不允许,相当于汉语``不可以······''。
\begin{itemize}
  \item 忙しくても、親に手紙を書くのを忘れてはならない。
\end{itemize}

{\bf
\noindent 动词连用形+「ていく」
}

表示动作、状态的发展趋势,相当于汉语``······起来''。
\begin{itemize}
  \item 少しずつ慣れていく。
\end{itemize}

{\bf
\noindent 动词连用形+「てしまう」
}

表示动作的完成。
\begin{itemize}
  \item 作文はもう書いてしまいました。
\end{itemize}

{\bf
\noindent 动词连用形+「ている」+「ところだ」
}

表示正在进行某一动作,相当于汉语``正在······''。
\begin{itemize}
  \item みんなは今、食事をしているところです。
\end{itemize}

{\bf
\noindent 动词连用形+「つつある」
}

表示动作持续进行,可用「ている」替换,用于书面。
\begin{itemize}
  \item 地球の環境はどんどん汚染されつつある。
\end{itemize}

{\bf
\noindent 动词连用形+「た」+「ところだ」
}

表示刚刚完成某一动作,相当于汉语``刚刚······''。
\begin{itemize}
  \item 飛行機は今、飛び立ったところです。
\end{itemize}

{\bf
\noindent 动词连用形+「てみる」
}

表示尝试做某事,相当于汉语``试着······''。
\begin{itemize}
  \item 自分で調べてみてください。
\end{itemize}

{\bf
\noindent 动词连用形+「てはいられない」
}

表示动作主体不能保持原来的状态,相当于汉语``不能······''。
\begin{itemize}
  \item もうこれ以上黙ってはいられない。
\end{itemize}

{\bf
\noindent 动词连用形+「た以上」
}

相当于汉语``既然······''。
\begin{itemize}
  \item 約束したた以上、守らなければならない。
\end{itemize}


\subsubsection{终止形}%

\begin{enumerate}
  \item 结句。
  \item 后接助动词。
  \item 后接助词。
\end{enumerate}

{\bf
\noindent 体言、用言终止形+かもしれません
}

表示对事物的估计,相当于汉语``也许······''。
\begin{itemize}
  \item 雨が降るかもしれません。
\end{itemize}

{\bf
\noindent 体言、用言终止形+に違いありません
}

表示确信,相当于汉语``一定是······''。
\begin{itemize}
  \item 彼はきっと成功するに違いありません。
\end{itemize}

{\bf
\noindent 用言终止形、体言+「とはいえ」
}

相当于汉语``虽然······,但是''。
\begin{itemize}
  \item 春とはいえ、まだ風が冷たい。
\end{itemize}

{\bf
\noindent 用言终止形、体言+「かどうか」
}

表示没有把握,相当于汉语``是否······''。
\begin{itemize}
  \item あの人が日本人かどうか私は知りません。
  \item 明日は雨が降るかどうかわかりません。
\end{itemize}


\subsubsection{连体形}%

\begin{enumerate}
  \item 后接体言,构成定语。
    \begin{itemize}
      \item 来る必要はありません。
    \end{itemize}
  \item 后接形式名词「こと」、「もの」等,将动词名词化。
    \begin{itemize}
      \item 早く起きることはいいことです。
    \end{itemize}
  \item 后接接续助词「ので」、「のに」等。
\end{enumerate}

{\bf
\noindent 体言の、用言连体形+ほうがいいです
}

多用于建议对方采取某种行为,相当于汉语``还是······为好''。
\begin{itemize}
  \item 明日は早く出発するから、早く寝たほうがいいです。
\end{itemize}

{\bf
\noindent 动词连体形+「に従って」
}

相当于汉语``随着······''。
\begin{itemize}
  \item 新しい技術の開発が進歩に従って、生産規模も拡大した。
\end{itemize}

{\bf
\noindent 动词连体形、体言+「につれて」
}

相当于汉语``随着······''。
\begin{itemize}
  \item 寮生活に慣れるにつれて、よく眠れるようになった。
\end{itemize}

{\bf
\noindent 动词连体形+「ところだ」
}

表示将要进行某一动作,相当于汉语``正要······''。
\begin{itemize}
  \item 今、出掛けるとこです。
\end{itemize}

{\bf
\noindent 动词连体形+「ことにする」
}

表示动作主体的主观决定,相当于汉语``决定······''。
\begin{itemize}
  \item 今度の夏休みに日本に旅行することにした。
\end{itemize}

{\bf
\noindent 动词连体形+「ことになる」
}

表示客观的决定、规定或自然产生的结果。
\begin{itemize}
  \item 彼は急に用事ができたので、私が一人で行くことになりました。
\end{itemize}


\subsubsection{假定形}%

后接接续助词「ば」,表示假定条件。
\begin{itemize}
  \item 明日雨が降れば、遠足をやめましょう。
\end{itemize}


\subsubsection{命令形}%

位于句末,表示命令。
\begin{itemize}
  \item もう一度読め。
  \item ただちに起きろ。
\end{itemize}



%%%%%%%%%%%%%%%%%%%%%%%%%%%%
\subsection{五段动词的音便}%
%%%%%%%%%%%%%%%%%%%%%%%%%%%%

五段动词的连用形在后接「て」、「ても」、「た」和「たり」时,
活用词尾要变成イ音便(カ行、ガ行)、促音便(タ行、ラ行、ワ行)和拨音便(ナ行、マ行、バ行)。

\begin{table}[h]
  \centering
  \caption{五段动词音便示例}
  \begin{tabular}{c | c | c | c c c c c c}
    分类 & 行 & 词例 & 词干 &  音便形 & 后续词 \\
    \hline
    \multirow{2}{*}{イ音便}
    & カ行 & 書く & \ruby{書}{か} & 書い & 書いて \\
    & ガ行 & 泳ぐ & \ruby{泳}{およ} & 泳い & 泳いで \\
    \hline
    \multirow{3}{*}{促音便}
    & タ行 & 立つ & \ruby{立}{た} & 立っ   & 立って \\
    & ラ行 & 取る & \ruby{取}{と} & 取っ   & 取って \\
    & ワ行 & 歌う & \ruby{歌}{うた} & 歌っ & 歌って \\
    \hline
    \multirow{3}{*}{拨音便}
    & ナ行 & 死ぬ & \ruby{死}{し} & 死ん & 死んで \\
    & マ行 & 読む & \ruby{読}{よ} & 読ん & 読んで \\
    & バ行 & 飛ぶ & \ruby{飛}{と} & 飛ん & 飛んで \\
  \end{tabular}
\end{table}



%%%%%%%%%%%%%%%%%%%%%%
\subsection{动词的态}%
%%%%%%%%%%%%%%%%%%%%%%

描述一件事情时,既可以从动作的施事者出发,
也可以从动作的受事者出发,
还可以从指使者出发。
由于动词所指动作与主语的关系所形成的谓语动词的形态变化,
叫做动词的``态''。
日语中动词的态有五类:主动态、被动态、使动态、可能态、自然发生态。

\subsubsection{被动态}%

当描述某一动作时,以承受者为主角进行描述,
该动词所表现的形态,就是被动态。

被动态的构成:
\begin{itemize}
  \item 五段动词未然形 \cn[1] + 「れる」
  \item 五段以外动词未然形+「られる」。
    其中サ变动词未然形+「られる」约音为「される」。
\end{itemize}

被动句中,谓语为动词被动态。
主语为动作的承受者或受害者,用「が」表示。
施事者以「--に」、「--から」等补语出现。

直接被动句,由他动词构成,可以还原为主动句:
\begin{itemize}
  \item 純子さんは先生に褒められた。
\end{itemize}

间接被动句,用自动词或他动词构成。
间接被动句中,
动词并没有直接作用于叙述的焦点,
而是句中的主语从动作施事者那里遭受到不利影响。
这里又分为两种情况,
一种是叙述的焦点与动作的施事者或承受者有必然联系:
\begin{itemize}
  \item 太郎はお母さんに日記を読まれた。
  \item 彼は二歳の時、両親に死なれな。
\end{itemize}
一种是两者没有必然联系,而是在叙述客观事实时偶然建立起了联系:
\begin{itemize}
  \item 私は帰る途中、雨に降られた。
\end{itemize}


\subsubsection{使动态}%

当描述某一动作时,以指使者、容许者为主角进行描述,
该动词所表现的形态,就是使动态。

使动态的构成:
\begin{itemize}
  \item 五段动词未然形 \cn[1] + 「せる」
  \item 五段以外动词未然形+「させる」。
    其中サ变动词未然形+「させる」约音为「される」。
\end{itemize}

使动句中,谓语为动词使动态。
主语是指使者,用「が」表示。
施事者以「--に」、「--を」等补语出现。

他动词的使动对象都用「--に」:
\begin{itemize}
  \item 先生が彼に論文を書かせる。
\end{itemize}

自动词的使动对象,表示强制性意思时用「--を」:
\begin{itemize}
  \item 子どもを9時までに寝させる。
\end{itemize}
表示非强制意思时用「--に」:
\begin{itemize}
  \item 小さい子どもに一人で大通りを渡らせるのは危ない。
\end{itemize}

被使动态,是使动态的被动态,描述的焦点是被指使者。


\subsubsection{可能态}%

当叙述焦点由动作本身转移到实现动作的能力或可能性时,
动作所发生的形态变化,称为可能态。

可能态的构成:
\begin{itemize}
  \item 一段动词+「られる」
  \item サ变动词词干+「できる」
  \item 动词连体形+「ことができる」
  \item 五段动词变为对应行的下一段动词
\end{itemize}

可能态的意义:
\begin{itemize}
  \item 表示能力:私は日本語お話すことがでけます。
  \item 表示可能性:用事があって出席できません。
  \item 表示许可:授業中は、大声で話すことができない。
\end{itemize}


\subsubsection{自然发生态}%

描述的焦点在于动作行为本身是自然而然的发生时,
动词的形态变化,就是自然发生态。

自然发生态的构成:
\begin{itemize}
  \item 五段动词未然形+「れる」
  \item 五段以外动词未然形+「れる」
\end{itemize}

自然发生态的意义:表示不由自主,
常用在人的感情、感觉、直觉、判断、思考的方面。
\begin{itemize}
  \item 日本料理は色が綺麗で、まるで芸術作品のように思われる。
  \item 雪が降ると、故郷のことが思い出される。
\end{itemize}



%%%%%%%%%%%%%%%%%%%%%%
\subsection{授受動詞}%
%%%%%%%%%%%%%%%%%%%%%%

授受动词共有7个,分为3组:
\begin{itemize}
  \item 他方授我方:「くれる」(尊敬语「くださる」)
  \item 我方授他方:「やる」、「あげる」(谦让语「差し上げる」)
  \item 我方受他方:「もらう」(谦让语「いただく」)
\end{itemize}

例句:
\begin{itemize}
  \item 友達が(私に)辞書をくれた。
  \item (私は)友達に辞書をもらった。
  \item 社長が弟にネクタイをくださった。
  \item 弟が社長にネクタイをいただいた。
  \item 兄が弟に500$やった。
  \item 兄が友達に時計おあげた。
  \item 兄が社長にお酒お差し上げた。
\end{itemize}


\subsubsection{作为补助动词的授受动词}%

动词连用形+「て」+授受动词,表示人物之间行为的往来与恩惠关系。
\begin{itemize}
  \item 駅まで送ってくれました。
  \item いろいろと手伝ってくれました。
  \item 弟の勉強を手伝ってやりました。
  \item 何もしてあげませんでした。
  \item この本を貸してあげましょう。
  \item 中村さんに絵を書いてもらいました。
\end{itemize}



