\section{動詞}%


%%%%%%%%%%%%%%%%%%%%%%%%
\subsection{动词的分类}%
%%%%%%%%%%%%%%%%%%%%%%%%

\noindent 根据活用规律:五段动词、一段动词、サ变动词、カ变动词。

\noindent 根据是否要求宾语:他动词、自动词。

\noindent 根据后接「ている」的情况:继续动词、瞬间动词、状态动词、形容词性动词。



%%%%%%%%%%%%%%%%%%%%%%%%%%
\subsection{动词的活用形}%
%%%%%%%%%%%%%%%%%%%%%%%%%%

\begin{table}[h]
  \centering
  \caption{动词活用示例}
  \scriptsize
  \begin{tabular}{c | c | c | c c c c c c}
    分类 & 词例 & 词干 & 未然形 & 连用形 & 终止形 & 连体形 & 假定形 & 命令形 \\
    \hline
    \multirow{9}{*}{五段}
    & 書く & \ruby{書}{か}   & \makecell{\cn[1] か \\ \cn[2] こ} & き & く & く & け & け \\
    & 泳ぐ & \ruby{泳}{およ} & \makecell{\cn[1] が \\ \cn[2] ご} & ぎ & ぐ & ぐ & げ & げ \\
    & 話す & \ruby{話}{はな} & \makecell{\cn[1] さ \\ \cn[2] そ} & し & す & す & せ & せ \\
    & 立つ & \ruby{立}{た}   & \makecell{\cn[1] た \\ \cn[2] と} & ち & つ & つ & て & て \\
    & 取る & \ruby{取}{と}   & \makecell{\cn[1] ら \\ \cn[2] ろ} & り & る & る & れ & れ \\
    & 歌う & \ruby{歌}{うた} & \makecell{\cn[1] わ \\ \cn[2] お} & い & う & う & え & え \\
    & 死ぬ & \ruby{死}{し}   & \makecell{\cn[1] な \\ \cn[2] の} & に & ぬ & ぬ & ね & ね \\
    & 飛ぶ & \ruby{飛}{と}   & \makecell{\cn[1] ば \\ \cn[2] ぼ} & び & ぶ & ぶ & べ & べ \\
    & 読む & \ruby{読}{よ}   & \makecell{\cn[1] ま \\ \cn[2] も} & み & む & む & め & め \\
    \hline
    \multirow{2}{*}{上一段}
    & 起きる & \ruby{起}{お} & き & き & きる & きる & きれ & \makecell{きろ\\きよ} \\
    & 見る & \ruby{見}{み}   & み & み & みる & みる & みれ & \makecell{みろ\\みよ} \\
    \hline
    \multirow{2}{*}{下一段}
    & 食べる & \ruby{食}{た} & べ & べ & べる & べる & べれ & \makecell{べろ\\べよ} \\
    & 寝る & \ruby{寝}{ね}   & ね & ね & ねる & ねる & ねれ & \makecell{ねろ\\ねよ} \\
    \hline
    \multirow{2}{*}{サ变动词}
    & する &  & \makecell{し\\せ} & し & する & する & すれ & \makecell{しよ\\せよ} \\
    & 勉強する & \ruby{勉強}{べんきょう} & \makecell{し\\せ} & し & する & する & すれ & \makecell{しよ\\せよ} \\
    \hline
    カ变动词 & 来る & & \ruby{来}{こ} & \ruby{来}{き} & \ruby{来}{く}る & \ruby{来}{く}る & \ruby{来}{く}れ & \ruby{来}{こ}い \\
  \end{tabular}
\end{table}

{\bf
\noindent 未然形
}

不能独立出现,必须后续助动词,才可表达完整的语法意义。
未然形后主要接一下几种助动词:
\begin{itemize}
  \item 「ない」,表示否定
  \item 「られる」等,构成不同的态
\end{itemize}

{\bf
\noindent 连用形
}

充当名词或其他词素,构成复合词。
\begin{itemize}
  \item 蒸し暑い
\end{itemize}

表示中顿。

后接助动词「ます」、「た」、「たい」、「そうだ」等。

后接助词「に」、「て」、「たり」等。

{\bf
\noindent 终止形
}

结句。

后接助动词。

后接助词。

{\bf
\noindent 连体形
}

后接体言,构成定语。
\begin{itemize}
  \item 来る必要はありません。
\end{itemize}

后接形式名词「こと」、「もの」等,将动词名词化。
\begin{itemize}
  \item 早く起きることはいいことです。
\end{itemize}

后接接续助词「ので」、「のに」等。

{\bf
\noindent 假定形
}

后接接续助词「ば」,表示假定条件。
\begin{itemize}
  \item 明日雨が降れば、遠足をやめましょう。
\end{itemize}

{\bf
\noindent 命令形
}

位于句末,表示命令。


%%%%%%%%%%%%%%%%%%%%%%%%%%%%
\subsection{五段动词的音便}%
%%%%%%%%%%%%%%%%%%%%%%%%%%%%

五段动词的连用形在后接「て」、「ても」、「た」和「たり」时,
活用词尾要变成イ音便(カ行、ガ行)、促音便(タ行、ラ行、ワ行)和拨音便(ナ行、マ行、バ行)。

\begin{table}[h]
  \centering
  \caption{五段动词音便示例}
  \begin{tabular}{c | c | c | c c c c c c}
    分类 & 行 & 词例 & 词干 &  音便形 & 后续词 \\
    \hline
    \multirow{2}{*}{イ音便}
    & カ行 & 書く & \ruby{書}{か} & 書い & 書いて \\
    & ガ行 & 泳ぐ & \ruby{泳}{およ} & 泳い & 泳いで \\
    \hline
    \multirow{3}{*}{促音便}
    & タ行 & 立つ & \ruby{立}{た} & 立っ   & 立って \\
    & ラ行 & 取る & \ruby{取}{と} & 取っ   & 取って \\
    & ワ行 & 歌う & \ruby{歌}{うた} & 歌っ & 歌って \\
    \hline
    \multirow{3}{*}{拨音便}
    & ナ行 & 死ぬ & \ruby{死}{し} & 死ん & 死んで \\
    & マ行 & 読む & \ruby{読}{よ} & 読ん & 読んで \\
    & バ行 & 飛ぶ & \ruby{飛}{と} & 飛ん & 飛んで \\
  \end{tabular}
\end{table}



%%%%%%%%%%%%%%%%%%%%%%
\subsection{动词的态}%
%%%%%%%%%%%%%%%%%%%%%%

描述一件事情时,既可以从动作的施事者出发,
也可以从动作的受事者出发,
还可以从指使者出发。
由于动词所指动作与主语的关系所形成的谓语动词的形态变化,
叫做动词的``态''。
日语中动词的态有五类:主动态、被动态、使动态、可能态、自然发生态。

\subsubsection{可能态}%

当叙述焦点由动作本身转移到实现动作的能力或可能性时,
动作所发生的形态变化,称为可能态。

可能态的构成:
\begin{itemize}
  \item 一段动词+「られる」
  \item サ变动词词干+「できる」
  \item 动词连体形+「ことができる」
  \item 五段动词变为对应行的下一段动词
\end{itemize}

可能态的意义:
\begin{itemize}
  \item 表示能力:私は日本語お話すことがでけます。
  \item 表示可能性:用事があって出席できません。
  \item 表示许可:授業中は、大声で話すことができない。
\end{itemize}


%%%%%%%%%%%%%%%%%%%%%%
\subsection{授受動詞}%
%%%%%%%%%%%%%%%%%%%%%%

授受动词共有7个,分为3组:
\begin{itemize}
  \item 他方授我方:「くれる」(尊敬语「くださる」)
  \item 我方授他方:「やる」、「あげる」(谦让语「差し上げる」)
  \item 我方受他方:「もらう」(谦让语「いただく」)
\end{itemize}

例句:
\begin{itemize}
  \item 友達が(私に)辞書をくれた。
  \item (私は)友達に辞書をもらった。
  \item 社長が弟にネクタイをくださった。
  \item 弟が社長にネクタイをいただいた。
  \item 兄が弟に500$やった。
  \item 兄が友達に時計おあげた。
  \item 兄が社長にお酒お差し上げた。
\end{itemize}


\subsubsection{作为补助动词的授受动词}%

动词连用形+「て」+授受动词,表示人物之间行为的往来与恩惠关系。
\begin{itemize}
  \item 駅まで送ってくれました。
  \item いろいろと手伝ってくれました。
  \item 弟の勉強を手伝ってやりました。
  \item 何もしてあげませんでした。
  \item この本を貸してあげましょう。
  \item 中村さんに絵を書いてもらいました。
\end{itemize}



